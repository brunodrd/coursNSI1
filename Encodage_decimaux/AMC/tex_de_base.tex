\documentclass[10pt,a4paper]{article}
\usepackage{listings}
%\usepackage{minted}
\usepackage[utf8]{inputenc}
\usepackage[T1]{fontenc}
\usepackage{amsmath}
\usepackage[francais,bloc,completemulti]{automultiplechoice}

\begin{document}

\exemplaire{38}{

%%% debut de l'entête des copies :    
	%%% première ligne

	\begin{minipage}{1\linewidth}
	  \centering\large\bf CODAGE DE NOMBRES 
	\end{minipage}

	%%% codage des numéros d'étudiants sur 3 chiffres par les candidats et encadré d'écriture manuelle des noms.
	\vspace*{3mm}
	{\setlength{\parindent}{0pt}\AMCcode{nsi}{2}\hspace*{\fill}
	\begin{minipage}[b]{12cm}
	 \hspace*{5mm}
  	  Codez votre numéro d'identification ci contre chiffre par chiffre,
  	  
  	  puis complétez l'encadré.
     \vspace{3ex}

	 \hfill\champnom{\fbox{
	 \begin{minipage}{1\linewidth}
		\vspace*{3mm}
		NOM - Prénom - Classe :
		\vspace*{3mm}
	 \end{minipage}
	    }
	   }
	\hfill\vspace{1ex}
% texte de présentation 
\begin{center}\em

Durée : 55 minutes.

  Document écrit non autorisé. Calculatrice autorisée.
  Les réponses fausses ou incohérentes retirent des points.

\end{center}

	\end{minipage}\hspace*{\fill}
	}

\vspace{4ex}

%%% Fin de l'entête
%%%%%%%%%%%%%%%%%%%%%%%


\element{codageentiernaturel}{
  \begin{question}{entiernaturel-Q1}\bareme{b=2,m=-0.25,e=-0.25,v=0}
  L'entier naturel $25$ s'écrit en binaire naturel sur 8 bits:    
   \begin{reponseshoriz}
   		\mauvaise{$0001 1010$}
   		\mauvaise{$0001 1000$}
   		\mauvaise{$0001 0101$}
   		\bonne{$0001 1001$}
   \end{reponseshoriz}
  \end{question}
}
%
%
\element{codageentiernaturel}{
  \begin{question}{entiernaturel-Q2}\bareme{b=2,m=-0.25,e=-0.25,v=0}
         On effectue l'addition binaire $0010 1101 + 0000 1011$. Quel est le résultat?
          \begin{reponseshoriz}
   			\mauvaise{$0011 1100$}
   			\mauvaise{$0010 1000$}
   			\mauvaise{$0010 0110$}
   			\bonne{$0011 1000$}
   \end{reponseshoriz}
  \end{question}
}
%
% 
\element{codageentiernaturel}{
  \begin{question}{entiernaturel-Q3}\bareme{b=2,m=-0.25,e=-0.25,v=0}
Quelle est la valeur de l'entier positif codé par le motif binaire $0001 1010$ ?

     \begin{reponseshoriz}
      \mauvaise{$22$}
      \bonne{$51$}
      \bonne{$26$}
      \mauvaise{$24$}
    \end{reponseshoriz} 
  \end{question}
}
%
%
\element{codageentiernaturel}{
  \begin{question}{entiernaturel-Q4}\bareme{b=2,m=-0.25,e=-0.25,v=0} 
	Quelle est la représentation sur 8 bits de l'addition binaire $1011 1011 + 0111 0101$?
     \begin{reponseshoriz}
      \mauvaise{Impossible}
      \bonne{$0011 0000$}
      \mauvaise{$1 0011 0000$}
      \mauvaise{$0000 0000$}
    \end{reponseshoriz} 
  \end{question} 
}
%
%
\element{codageentiernaturel}{
  \begin{question}{entiernaturel-Q5}\bareme{b=2,m=-0.25,e=-0.25,v=0}
On considère le nombre $N=1000_{10}$ (\emph{écrit en base 10}). L'écriture de $N$ en binaire:
     \begin{reponses}
      \mauvaise{comporte 4 chiffres}
      \bonne{comporte au moins 9 chiffres}
      \mauvaise{se termine par 1}
      \mauvaise{comporte moins de 9 chiffres}
    \end{reponses} 
  \end{question} 
}


\element{codageentierrelatif}{
  \begin{question}{entierrelatif-Q1} \bareme{b=2,m=-0.25,e=-0.25,v=0}
  Quelle est la représentation de $-3$ sur 8 bits, par la méthode du complément à deux?
     \begin{reponseshoriz}
      \mauvaise{$0000 0101$}
      \bonne{$1111 1101$}
      \mauvaise{$1111 1100$}
      \mauvaise{$0000 0100$}
    \end{reponseshoriz} 
  \end{question}
}
%
%
\element{codageentierrelatif}{
  \begin{question}{entierrelatif-Q2} \bareme{b=2,m=-0.25,e=-0.25,v=0}
 Que vaut le nombre binaire $1110 0000$ codé par la méthode du complément à deux, sur 8 bits?
     \begin{reponseshoriz}
      \mauvaise{$224$}
      \bonne{$-32$}
      \mauvaise{$-224$}
      \mauvaise{$-96$}
    \end{reponseshoriz} 
  \end{question}
}
%
%
\element{codageentierrelatif}{
  \begin{question}{entierrelatif-Q3} \bareme{b=2,m=-0.25,e=-0.25,v=0}
 On travaille avec des entiers relatifs codés sur 8 bits. L'addition binaire ${0111 1111 + 0000 0001}$:
     \begin{reponses}
      \mauvaise{donne un nombre positif}
      \bonne{donne un nombre négatif}
      \mauvaise{zéro}
      \mauvaise{est impossible}
    \end{reponses} 
  \end{question}
  }
 %
 %
  \element{codageentierrelatif}{
  \begin{question}{entierrelatif-Q4} \bareme{b=2,m=-0.25,e=-0.25,v=0}
	Quel est le codage de l'entier relatif positif 64 sur 8 bits?
     \begin{reponseshoriz}
      \mauvaise{$1100 0000$}
      \bonne{$0100 0000$}
      \mauvaise{$1110 0000$}
      \mauvaise{$0110 0000$}
    \end{reponseshoriz} 
  \end{question}
  }
  %
  %
  \element{codageentierrelatif}{
  \begin{question}{entierrelatif-Q5} \bareme{b=2,m=-0.25,e=-0.25,v=0}
Le nombre binaire $0111 1111$ codé sur 8 bits est:
     \begin{reponses}
      \mauvaise{le plus petit entier relatif négatif qu'on peut coder sur 8 bits}
      \bonne{le plus grand entier relatif positif qu'on peut coder sur 8 bits}
      \mauvaise{le codage de zéro}
      \mauvaise{est un cas particulier: il a la même représentation que son opposé}
    \end{reponses} 
  \end{question}
  }
  %
  %
    \element{codageentierrelatif}{
  \begin{question}{entierrelatif-Q6} \bareme{b=2,m=-0.25,e=-0.25,v=0}
	Dans une représentation d'entiers relatifs sur 8 bits par la méthode du complément à deux, le bit de signe est:
     \begin{reponses}
      \mauvaise{le bit de poids faible (\emph{bit 0})}
      \bonne{le bit de poids fort (\emph{bit 7})}
      \mauvaise{obtenu en ajoutant $1$ au nombre}
      \mauvaise{obtenu en inversant les bits}
    \end{reponses} 
  \end{question}
  }
%
%
     \element{codageentierrelatif}{
  \begin{question}{entierrelatif-Q7} \bareme{b=2,m=-0.25,e=-0.25,v=0}
	La méthode du complément à deux permet:
     \begin{reponses}
      \mauvaise{d'inverser tous les bits d'un nombre entier écrit en binaire}
      \bonne{d'obtenir l'opposé d'un nombre entier écrit en binaire}
      \mauvaise{d'ajouter un à un nombre entier écrit en binaire}
      \mauvaise{de trouver la valeur absolue d'un entier relatif}
    \end{reponses} 
  \end{question} 
  }
  %
  %
   \element{codageentierrelatif}{
  \begin{question}{entierrelatif-Q8} \bareme{b=2,m=-0.25,e=-0.25,v=0}
	Quelle est la valeur de l'entier relatif dont la représentation en binaire est $0111 1110$ (\emph{sur 8 bits})?
     \begin{reponseshoriz}
      \mauvaise{$-126$}
      \bonne{$126$}
      \mauvaise{$-124$}
      \mauvaise{$-128$}
    \end{reponseshoriz} 
  \end{question}
 }
 %
 %
    \element{codagereels}{
  \begin{question}{codagereels-Q1} \bareme{b=2,m=-0.25,e=-0.25,v=0}
	Quel est le codage en binaire selon la méthode de la virgule fixe, du nombre réel $3,25$?
     \begin{reponseshoriz}
      \mauvaise{$11,11001$}
      \bonne{$11,01$}
      \mauvaise{$3,11001$}
      \mauvaise{$1,101$}
    \end{reponseshoriz} 
  \end{question}
  }
  %
  %
     \element{codagereels}{
  \begin{question}{codagereels-Q2} \bareme{b=2,m=-0.25,e=-0.25,v=0}
	Que vaut le nombre binaire $10001,01$ codé selon la méthode de la virgule fixe?
     \begin{reponseshoriz}
      \mauvaise{$17,01$}
      \bonne{$17,25$}
      \mauvaise{$17,1$}
      \mauvaise{$1,000101$}
    \end{reponseshoriz} 
  \end{question}
  }
  %
  % 
     \element{codagereels}{
  \begin{question}{codagereels-Q3} \bareme{b=2,m=-0.25,e=-0.25,v=0}
	Le nombre $10010,0011$ peut s'écrire:
     \begin{reponseshoriz}
      \mauvaise{$1,00100011\times 2^{-4}$}
      \bonne{$1,00100011\times 2^4$}
      \mauvaise{$1,00100011\times 10^4$}
      \mauvaise{$1,00100011\times 10^{-4}$}
    \end{reponseshoriz} 
  \end{question}
  }
  %
  %   
     \element{codagereels}{
  \begin{question}{codagereels-Q4} \bareme{b=2,m=-0.25,e=-0.25,v=0}
	La représentation en virgule flottante est écriture du type \emph{signe}\vert \emph{exposant}\vert \emph{mantisse}. Que vaut le nombre ${0\ 10000011\ 10010100000000000000000}$ codé en simple précision (\emph{32 bits})?
     \begin{reponseshoriz}
      \mauvaise{$131,578125$}
      \bonne{$25,25$}
      \mauvaise{$0,578125\times 2^{131}$}
      \mauvaise{$9,25$}
    \end{reponseshoriz} 
  \end{question}
  }
   %
  %   
     \element{codagereels}{
  \begin{question}{codagereels-Q5} \bareme{b=2,m=-0.25,e=-0.25,v=0}
	Quelle est la représentation en virgule flottante, simple précision de $-132,5$?
     \begin{reponses}
      \mauvaise{$0\ 10000110\ 00001001000000000000000$}
      \bonne{$1\ 10000110\ 00001001000000000000000$}
      \mauvaise{$1\ 10000100\ 10000000000000000000000$}
      \mauvaise{$1\ 10000111\ 00001001000000000000000$}
    \end{reponses} 
  \end{question}
  }
   %
  %   
     \element{codagereels}{
  \begin{question}{codagereels-Q6} \bareme{b=2,m=-0.25,e=-0.25,v=0}
	L'opération \texttt{0.1*12} en python fournit \texttt{1.2000000000000002}. Quelle en est la raison?
     \begin{reponses}
      \mauvaise{La calculatrice de python est plus précise qu'une calculatrice ordinaire}
      \bonne{Les nombres réels sont représentés de manière approximative en machine}
      \mauvaise{L'opérateur aurait dû saisir \texttt{float(0.1*12}}
      \mauvaise{Par défaut tous les calculs sur les décimaux sont fourni avec 16 décimales}
    \end{reponses} 
  \end{question}
  }
    %
  %   
     \element{codagereels}{
  \begin{question}{codagereels-Q7} \bareme{b=2,m=-0.25,e=-0.25,v=0}
	Cochez une propriété correcte des nombres flottants sur une machine numérique.
     \begin{reponses}
      \mauvaise{La représentation en virgule flottante nécessite 3 octets pour coder le signe, l'exposant et la mantisse}
      \bonne{Des propriétés mathématiques comme l'associativité de l'addition ne sont pas forcément valables avec les flottants}
      \mauvaise{Pour coder un nombre réel négatif on utilise la méthode du complément à deux}
      \mauvaise{Il n'est pas possible de coder zéro avec la norme IEEE754 qui définit les règles de codage et d'utilisation des flottants}
    \end{reponses} 
  \end{question}
  }   
 %
 % 
     \element{codagereels}{
  \begin{question}{codagereels-Q8} \bareme{b=2,m=-0.25,e=-0.25,v=0}
	L'instruction \texttt{0.1 + 0.2 == 0.3} en python, fournira:
     \begin{reponseshoriz}
      \mauvaise{\texttt{True}}
      \bonne{\texttt{False}}
      \mauvaise{\texttt{0.3}}
      \mauvaise{\texttt{SyntaxError}}
    \end{reponseshoriz} 
  \end{question}
  }
  
\AMCcleardoublepage

}

\end{document}

