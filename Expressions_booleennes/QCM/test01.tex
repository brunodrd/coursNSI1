\documentclass[10pt,a4paper]{article}

\usepackage[utf8]{inputenc}
\usepackage[T1]{fontenc}

\usepackage[francais,bloc,completemulti]{automultiplechoice}
\begin{document}

\exemplaire{10}{

%%% debut de l’en-tête des copies :

\noindent{\bf QCM \hfill SPECIALIT\'E NSI}

\vspace*{.5cm}
\begin{minipage}{.4\linewidth}
\centering\large\bf LOGIQUE BOOL\'EENNE\end{minipage}
\champnom{\fbox{
				\begin{minipage}{.5\linewidth}
					Nom et prénom :
				
					\vspace*{.5cm}\dotfill
					\vspace*{1mm}
				\end{minipage}
		}}
		
\begin{center}\em
Durée : 10 minutes.

L’usage de la calculatrice n'est pas autorisé.

Les questions ont une unique bonne réponse.

Des points négatifs pourront être affectés à de \emph{très
mauvaises} réponses.
\end{center}
\vspace{1ex}
%%% fin de l’en-tête


\begin{question}{operateurs}
Quels sont les trois opérateurs booléens fondamentaux?
	\begin{reponses}
		\bonne{NON, ET, OU}
		\mauvaise{OUI, NON, OU}
		\mauvaise{OUI, ET, PEUT-\^ETRE}
		\mauvaise{OUI, ET, OU}
	\end{reponses}
\end{question}

\begin{question}{nand}
On considère le circuit logique présenté à la figure 1  de la feuille annexe. Quelle table de vérité correspond à ce circuit?
	\begin{reponses}
		\bonne{Table 5}
		\mauvaise{Table 2}
		\mauvaise{Table 3}
		\mauvaise{Table 4}
		\mauvaise{Table 1}
	\end{reponses}
\end{question}

\begin{question}{aouabarre}
Soit $A$ une variable logique. On considère l'expression booléenne $S=A+\overline{A}$. Quelle proposition est correcte?
\begin{reponses}
		\bonne{On a toujours $S=1$}
		\mauvaise{On a toujours $S=0$}
		\mauvaise{On a toujours $S=A$}
		\mauvaise{On a toujours $S=\overline{A}$}
\end{reponses}
\end{question}

\begin{question}{aetabarre}
Soit $A$ une variable logique. On considère l'expression booléenne $S=A\cdot\overline{A}$. Quelle proposition est correcte?
\begin{reponses}
		\bonne{On a toujours $S=1$}
		\mauvaise{On a toujours $S=0$}
		\mauvaise{On a toujours $S=A$}
		\mauvaise{On a toujours $S=\overline{A}$}
\end{reponses}
\end{question}

\begin{question}{expressionnand}
On considère le circuit logique présenté à la figure 1  de la feuille annexe. \`A quelle expression booléenne  correspond la sortie $S$ ce circuit?
	\begin{reponses}
		\bonne{$\overline{A}+\overline{B}$}
		\mauvaise{$\overline{A}\cdot\overline{B}$}
		\mauvaise{$A+B$}
		\mauvaise{$A\cdot B$}
	\end{reponses}
\end{question}

\begin{question}{demiadd}
On considère un demi additionneur 1 bit. Les entrées sont notées $A$ et $B$, la sortie $S$ et la retenue $R$. Quelle proposition est correcte?
	\begin{reponses}
		\bonne{$S=A\oplus B$, $R=A\cdot B$}
		\mauvaise{$S=A\oplus B$, $R=A+ B$}
		\mauvaise{$S=A+ B$, $R=A\cdot B$}
		\mauvaise{$S=A\cdot B$, $R=A+ B$}
		\mauvaise{$S=A\cdot B$, $R=A\oplus B$}
	\end{reponses}
\end{question}
% \AMCaddpagesto{3}

}

\end{document}